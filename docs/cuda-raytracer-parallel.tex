% ----- formatovani dokumentu -----------------------------------------------
\documentclass[12pt,a4paper,titlepage,final]{report}
\usepackage[utf8]{inputenc}
\usepackage[T1, IL2]{fontenc}
\usepackage{graphicx}
\usepackage{epstopdf}
\usepackage[margin=2cm]{caption}
\usepackage[top=3cm, left=2cm, right=2cm, text={17cm, 24cm}, ignorefoot]{geometry}
\usepackage[usenames,dvipsnames]{color}
\usepackage[]{algorithm2e}
\usepackage{amsmath}

% ------ commands -----------------------


% ---------------------------------------

\usepackage{url}
\usepackage{setspace}
\singlespacing
\usepackage[square, numbers]{natbib} 
\pagestyle{plain}
\pagenumbering{arabic}
\setcounter{page}{1}

\setlength{\parindent}{1cm}	
\usepackage{natbib}
\renewcommand{\thesection}{\arabic{section}}
\renewcommand{\thesubsection}{\arabic{section}.\arabic{subsection}}



% ----- vyberte jazyk -------------------------------------------------------
\usepackage[english,czech]{babel}
%\usepackage[english]{babel}

% ----- dopiste titulky -----------------------------------------------------
\newcommand\Course{Grafické a multimediální procesory}
\newcommand\WorkTitle{Raytracer na CUDA}
\newcommand\AuthorA{Pavel Macenauer}
\newcommand\AuthorB{Jan Bureš}
\newcommand\AuthorAEmail{xmacen02@stud.fit.vutbr.cz}
\newcommand\AuthorBEmail{xbures19@stud.fit.vutbr.cz}
\newcommand\Faculty{Fakulta Informačních Technologií}
\newcommand\School{Vysoké Učení Technické v~Brně}

\usepackage[
pdftitle={\WorkTitle},
pdfauthor={\AuthorA\AuthorB},
bookmarks=true,
colorlinks=true,
breaklinks=true,
urlcolor=blue,
citecolor=blue,
linkcolor=blue,
unicode=true,
]
{hyperref}



% ----- titulni strana ------------------------------------------------------

\begin{document}
	\begin{titlepage}
	\begin{center}
		\includegraphics[height=5cm]{images/logo.eps}
	\end{center}
	\vfill
	\begin{center}
		\begin{Large}
			\Course\\
		\end{Large}
		\bigskip
		\begin{Huge}
			\WorkTitle\\
		\end{Huge}
	\end{center}
	\vfill
	\begin{center}
		\begin{large}
			\today
		\end{large}
	\end{center}
	\vfill
	\begin{flushleft}
		\begin{large}
			\begin{tabular}{lll}
				Autor: & \AuthorA, & \url{\AuthorAEmail} \\
				& \AuthorB, & \url{\AuthorBEmail} \\
		
				& & \\
				& \Faculty \\
				& \School \\
			\end{tabular}
		\end{large}
	\end{flushleft}
\end{titlepage}		

\tableofcontents

% ----- obsah -------------------------------------------------------------
\newpage

\section{Cíl práce}

Cílem tohoto projektu bylo vytvořit raytracer, který využije potenciálu grafických karet v oblasti paralelního zpracování algoritmu. 

\section{Raytracing}

\dots

\section{Optimalizace pro GPU}

\dots

\subsection{Paralelizace}

\dots

\subsection{Uspořádání paměti}

\dots

\subsection{Další možné zlepšení}

\dots

\section{Krátce co Raytracer umí}

\dots

\section{Rozdělení práce}

\paragraph{Pavel Macenauer} \dots

\paragraph{Jan Bureš} \dots

\section{Ovládání programu}

\subsection{Překlad}

\dots

\subsection{Spuštění a ovládání}

\dots

\section{Závěr}

\dots

\bibliographystyle{plain}

\nocite{cite1}
\nocite{cite2}
\nocite{cite3}
\nocite{cite4}


\hypertarget{bib}{}
\bibliography{reference}
\addcontentsline{toc}{section}{Literatura}

\end{document}

